\documentclass{article}
\usepackage{amsmath, amsfonts, amssymb, amstext, amscd, amsthm, fullpage, makeidx, graphicx, hyperref, url}

\newcommand{\unit}[1]{\mathrm{~#1}}

\title{Chem 1a PSet 1 Responses}
\author{Gavin Hua}
\date{September 2023}

\begin{document}

\maketitle

\section*{Question 1.}
\subsection*{(a)}
A single basketball game burns the following amount of calories:
$$
1\mathrm{~game}(\frac{112\mathrm{~length}}{1\mathrm{~game}})(\frac{94\mathrm{~feet}}{1\mathrm{~length}}) (\frac{1\mathrm{~km}}{3281\mathrm{~feet}}) (\frac{62\mathrm{~cal}}{1\mathrm{~km}}) = 198.942\mathrm{~cal~~(2SF)}.
$$
Swimming burns calories at the following rate:
$$
\frac{17\mathrm{~cal}}{1\mathrm{~lap}} \mathrm{~(2SF)}.
$$
Laps needed to burn the same number of calories as he would from a game of basketball:
$$
198.942\mathrm{~cal} (\frac{1\mathrm{~lap}}{17\mathrm{~cal}}) = 11.702 \mathrm{~lap} = 12 \mathrm{~lap(s)}.
$$
\subsection*{(b)}
The number of atoms that can be fit in between the lines is:
$$
(\frac{0.7\mathrm{~mm}}{170\mathrm{~pm} / 1 \mathrm{~atom}}) = (\frac{0.7\mathrm{~mm}}{170\mathrm{~pm} / 1 \mathrm{~atom}}) (\frac{10^{12} \mathrm{~pm}}{1 \mathrm{~m}}) (\frac{1 \mathrm{m}}{10^3 \mathrm{~mm}}) = 4.118 \cdot 10^6 \mathrm{~atom} = 4 \cdot 10^6 \mathrm{~atom(s)}.
$$
\subsection*{(c)}
The rotation video's length is:
$$
20\mathrm{~min} (\frac{60\mathrm{~s}}{1\mathrm{~min}}) + 39 \mathrm{~s} = 1239 \mathrm{~s~(4SF)}.
$$
She could watch the following number of TikTok videos:
$$
\frac{1239 \mathrm{~s}}{32.4 \cdot 10^9 \mathrm{~ns}} = (\frac{1239 \mathrm{~s}}{32.4 \cdot 10^9 \mathrm{~ns} / 1 \mathrm{~video}}) (\frac{10^9 \mathrm{~ns}}{1 \mathrm{~s}}) = 38.24 \mathrm{~video} = 38.24 \mathrm{~video} = 38.2 \mathrm{~video(s)}.
$$
\newpage

\section*{Question 2.}
\subsection*{(a)}
The total charge fired is:
$$
173\mathrm{~mA} \cdot 6.000 \mathrm{~h} = 173 \mathrm{~mA} (\frac{1\mathrm{~A}}{10^3 \mathrm{~mA}}) \cdot 6.000 \mathrm{~h} (\frac{3600\mathrm{~s}}{1\mathrm{~h}}) (\frac{1\mathrm{~C}}{1 \mathrm{~As}}) = 3736.8 \mathrm{~C} = 3.74\cdot10^3 \mathrm{~C}.
$$
\subsection*{(b)}
The total mass of collected electronics, in grams, is:
$$
(\frac{3736.8 \mathrm{~C}}{1.758820\cdot10^{11} \mathrm{~C~kg^{-1}}}) (\frac{1000\mathrm{~g}}{1\mathrm{~kg}}) = 2.1246\cdot10^{-5} \mathrm{~g} = 2.12\cdot10^{-5}\mathrm{~g}.
$$
\subsection*{(c)}
\subsubsection*{i.}
We assume a square lattice configuration for the gold atoms.
The number of gold atoms that can fit on one side of the gold film is:
$$
\frac{10.0\mathrm{~cm}}{2\cdot0.146\mathrm{~nm}} = (\frac{10.0\mathrm{~cm}}{2\cdot0.146\mathrm{~nm}}) (\frac{1\mathrm{~m}}{100\mathrm{~cm}})(\frac{10^9 \mathrm{~nm}}{1\mathrm{~m}}) = 3.42466\cdot10^8\mathrm{~(3SF)}.
$$
The total number of gold atoms is:
$$
(3.42466\cdot10^8)^2 = 1.1728\cdot10^{17} = 1.17\cdot10^{17}.
$$
\subsubsection*{ii.}
The total number of moles is:
$$
1.1728\cdot10^{17} \cdot \frac{1\mathrm{~mol}}{6.022\cdot10^{23}} = 1.9475\cdot10^{-7}\mathrm{~mol} = 1.95\cdot10^{-7}\mathrm{~mol}.
$$
\subsection*{(d)}
\subsubsection*{i.}
The total area of the gold film is:
$$
(10.0\unit{cm}\frac{1\unit{m}}{100\unit{cm}})^2 = 1.00\cdot10^{-2}\unit{m^2}.
$$
\subsubsection*{ii.}
The total area of all the nuclei is:
$$
1.1728\cdot10^{17}(\pi(7.0\cdot10^{-15}\unit{m)^2}) = 1.8053853\cdot10^{-11}\unit{m^2} = 1.81\cdot10^{-11}\unit{m^2}.
$$
\subsection*{(e)}
The probability is
$$
\frac{1.8053853\cdot10^{-11}\unit{m^2}}{1.00\cdot10^{-2}\unit{m^2}} = 1.81\cdot10^{-9}.
$$
This means that the size of the nucleus is minuscule when compared to the size of the atom.
\newpage

\section*{Question 3.}
The number of protons is $53,$ the number of neutrons is $78,$ the number of electrons is $53.$
\newpage

\section*{Question 4.}
\subsection*{(a)}
$$
\mathrm{
2Fe + 3O_2 \rightarrow 2Fe_2O_3
}
$$
\subsection*{(b)}
$$
\mathrm{
3C_2H_2O_4 + Fe_2O_3 \rightarrow Fe_2(C_2O_4)_3 + 3H_2O
}
$$
\subsection*{(c)}
$$
\mathrm{
6CH_3COOH + Fe_2O_3 \rightarrow 2Fe(CH_3COO)_3 + 3H_2O
}
$$
\subsection*{(d)}
$$
\mathrm{
C_6H_8O_7 + Fe_2O_3 \rightarrow 2FeO + 6CO + 2H_2 + 2H_2O
}
$$
\subsection*{(e)}
The mass of acetic acid in a bottle of vinegar is:
$$
16.0\unit{fl.oz.} (\frac{29.5735\unit{cm^3}}{1 \unit{fl.oz.}}) (\frac{1.006\unit{g}}{1\unit{cm^3}}) \cdot 5.00\% = 23.800\unit{g~(3SF)}.
$$
The mass of rust (molar mass $159.7$) that the acetic acid (molar mass $60.052$) can remove is:
$$
23.8\unit{g} (\frac{1\unit{mol}}{60.052\unit{g}}) (\frac{1\unit{mol}}{6\unit{mol}}) (\frac{159.7\unit{g}}{1\unit{mol}}) = 10.549\unit{g~(3SF)}
$$
The mass of citric acid in a bottle of lemon juice:
$$
4.00\unit{fl.oz.} (\frac{29.5735\unit{cm^3}}{1 \unit{fl.oz.}}) (\frac{1.01\unit{g}}{1\unit{cm^3}}) \cdot 7.00\% = 8.363\unit{g~(3SF)}.
$$
The mass of rust that the citric acid (molar mass $192.124$) can remove is:
$$
8.363\unit{g} (\frac{1\unit{mol}}{192.124\unit{g}}) (\frac{1\unit{mol}}{1\unit{mol}}) (\frac{159.7\unit{g}}{1\unit{mol}}) = 6.952\unit{g~(3SF)}
$$
Therefore, the bottle of vinegar can remove more rust. The difference is 
$$
10.549\unit{g} - 6.952\unit{g} = 3.597\unit{g} = 3.60\unit{g}
$$
\newpage

\section*{Question 5.}
\subsection*{(a)}
Let
$$
\mathrm{
a_1K_2Cr_2O_7 + a_2H_2SO_4 \rightarrow b_1K_2SO_4 + b_2Cr_2(SO_4)_3 + b_3H_2O + b_4O_2
}.
$$
The equations are
$$
\begin{cases}
    2a_1 &= 2b_1 \mathrm{~(K)}\\
    2a_1 &= 2b_2 \mathrm{~(Cr)}\\
    7a_1 + 4a_2 &= 4b_1  + 12b_2 + b_3 + 2b_4 \mathrm{~(O)}\\
    2a_2 &= 2b_2 \mathrm{~(H)}.\\
\end{cases}
$$
The relatively prime, integer solution is:
$$
\mathrm{
2K_2Cr_2O_7 + 8H_2SO_4 \rightarrow 2K_2SO_4 + 2Cr_2(SO_4)_3 + 8H_2O + 3O_2
}.
$$
\subsection*{(b)}
Let
$$
\mathrm{
a_1K_2Cr_2O_7 + a_2H_2SO_4 + a_3K_3(Fe(SCN)_6) \rightarrow
}
$$
$$
\mathrm{
b_1Fe_2(SO_4)_3 + b_2Cr_2(SO_4)_3 + b_3CO_2 + b_4H_2O + b_5K_2SO_4 + b_6KNO_3
}.
$$
The equations are
$$
\begin{cases}
    2a_1 + 3a_3 &= 2b_5 + b_6 \mathrm{~(K)}\\
    2a_1 &= 2b_2 \mathrm{~(Cr)}\\
    7a_1 + 4a_2 &= 12b_1  + 12b_2 + 2b_3 + b_4 + 4b_5 + 3b_6 \mathrm{~(O)}\\
    2a_2 &= 2b_4 \mathrm{~(H)}\\
    a_2 + 6a_3 &= 3b_1 + 3b_2 + b_5 + b_6 \mathrm{~(S)}\\
    a_3 &= 2b_1 \mathrm{~(Fe)}\\
    6a_3 &= b_6 \mathrm{~(N)}.\\
\end{cases}
$$
The relatively prime, integer solution is:
\[
\mathrm{
32K_2Cr_2O_7 + 116H_2SO_4 + 2K_3(Fe(SCN)_6) \rightarrow
}
\]
$$
\mathrm{
Fe_2(SO_4)_3 + 32Cr_2(SO_4)_3 + 12CO_2 + 116H_2O + 29K_2SO_4 + 12KNO_3
}.
$$
\newpage

\section*{Question 6.}
\subsection*{(a)}
Mass spectrometry measures individual atoms and molecules to determine their atomic and molecular weights. 
\subsection*{(b)}
We would be able to determine sources of atmospheric and water pollution.
\newpage
\end{document}
